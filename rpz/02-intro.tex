\chapter{Введение}
\label{cha: intro}
	Сегодня сложно представить себе работу с персональным компьютером без привычного каждому из нас устройства - мыши. С момента создания этот девайс претерпел колоссальные изменения. От обычных механических мышей, с которыми было неудобно работать в виду тяжелого веса и плохого позиционирования, мир перешел к оптическим, отличавшимся легкостью, надежностью и высокой точностью. Несмотря на это, метаморфозы устройства продолжаются. В настоящее время появились гироскопические мыши\cite{gyrmouse}, позволяющее распознавать движение не только на поверхности, но и в пространстве. Для тех, кто использует большие плазменные экраны или проекторы, такое новшество позволяет управлять компьютером в качестве пульта, не задействуя посторонние предметы. Но стоимость таких мышей составляет порядка 4-5 тысячи рублей, что в свою очередь является дорогим удовольствием.
	
\label{txt: tasks} 
% Из задач должно быть понятно, что я хочу сделать в курсовом проекте. Необходимо конкретезировать каждую задачу и иметь для нее ощутимую метрику. (Добавить точные цифры к требованиям)
	Данное программное обеспечение позволяет использовать Android-устройство в качестве беспроводной мыши и выполнять соответствующую функциональность. В процессе реализации должны быть решены следующие задачи:
\begin{itemize}
\item Задержка обработки данных.
	Программное обеспечение обрабатывает данные в реальном времени. При таком подходе важно учесть скорость передачи данных и их последующую обработку. Поскольку передается небольшой объем данных, со скоростью не должно возникнуть проблем, а вот слишком большая задержка увеличит дергание курсора.
\item Дрожание мыши.
	Для реализации используются датчики Android-устройства, погрешность которых достигает порядка 5° (в зависимости от выбора датчика). Помимо того, использование акселерометра или гироскопа добавляет внешний фактор - дрожание человеческой руки. Все это в совокупности вызывает дребезжание курсора, которое необходимо минимизировать.
\item Функциональность.
	Под функциональностью понимается минимальный набор требуемых функций, необходимых для замены базовой мыши. В данном случае будет достаточно поддержки двух осей вращения и двух кнопок.
\end{itemize}

%%% Local Variables: debug
%%% mode: latex
%%% TeX-master: "rpz-os"
%%% End