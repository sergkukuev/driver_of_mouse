\chapter{Технологический раздел}
\label{cha: impl}
\label{sec: lang}
\section{Выбор языка программирования}
	Комплекс программ разработан для использования в операционной системе Linux Ubuntu. В рамках реализации было разработано три программы:
\begin{itemize}
\item Драйвер устройства.
\item Bluetooth сервер.
\item Пользовательское приложение для Android устройства.
\end{itemize}
	Драйвер устройства для Linux разработан на языке Си. Данный выбор ограничен внутренним устройством ОС Linux и отсутствием средств разработки с использованием других языков.
	
	Bluetooth сервер для приема данных разработан на языке Python. Выбор основан на простоте использования rfcomm сокетов с использованием протокола sdp.
	
	Пользовательское приложение на платформе Android разработано на Java. Данный язык имеет строгую статическую типизацию, за счет чего выигрывает в производительности.
\label{sec: enviroment}
\section{Выбор среды программирования}
	Для драйвера устройства и bluetooth сервера было решено использовать стандартный текстовый редактор и соответствующие компиляторы для языков: C и Python - gcc и python. В качестве средства разработки был выбран Sublime Text 2 - текстовый редактор с подцветкой синтаксиса. Приложение под Android разрабатывалось в интегрированной среде разработке Android Studio. Выбор был сделан в пользу данной IDE, поскольку Eclipse прекратила поддержку плагина Android Development Tools(ADP) в 2014 году. Из плюсов необходимо выделить:
\begin{itemize}
\item Способность протестирвать приложение на устройствах с разным экраном и с разной версией API.
\item Множество шаблонов и макетов компонентов Android.
\end{itemize}
\label{sec: set&use}
\section{Установка и использование программного обеспечения}
	Для корректной работы драйвера и Android приложения необходимо скомпилировать модуль с помощью make файла и положить его в ядро с помощью команды \texttt{sudo insmod name\_module}. Далее необходимо включить bluetooth на компьютере и запустить сервер. После проделанных операций можно использовать Android приложение. Интерфейс приложения представлен на рисунке~\ref{fig: interface}.
	
\begin{figure} [h]
  \centering
  \includegraphics[scale=0.3]{impl/interface}
  \caption{Интерфейс приложения.}
  \label{fig: interface}
\end{figure}

\subsection*{Функционал Android приложения}
	Как видно на рисунке~\ref{fig: interface} сверху слева показывается информация о координатах x, y и z гиродатчика. Строкой ниже выводится информация о состоянии bluetooth и подключения к серверу:
\begin{itemize}
\item \texttt{Bluetooth disable} - bluetooth неактивен, нет соединения с сервером.
\item \texttt{Bluetooth enable} - bluetooth активен, нет соединения с сервером.
\item \texttt{Device\_name (device\_address)} - соединение с сервером активно, где Device\_name - имя сервера, device\_address - адресс сервера.
\end{itemize}

	Для взаимодействием с пользователем в программном обеспечении используются кнопки. В совокупности они составляют функционал программы, который представлен в таблице~\ref{tabl: function}.

\begin{table}[h]
  \caption{Функционал Android приложения}
  \begin{tabular}{|r|r|}
 	\hline
    Кнопка 		& 						Действие							\\
    \hline
    BT			&			Включение/выключение bluetooth					\\
    \hline
	Connect		&			Подключение к серверу							\\
    \hline
    Disconnect	&			Отключение от сервера							\\
    \hline
    Left Click	&  Нажатие левой кнопки мыши. Предусмотрено двойное нажатие.\\
    \hline
    Right Click	&			Нажатие правой кнопки мыши						\\
    \hline
  \end{tabular}
  \label{tabl: function}
\end{table} 
\label{sec: param}
\section{Технические требования}

	Требования для программного обеспечения минимальны. Для запуска драйвера необходима операционная система Linux. Для запуска приложения на Android устройстве требуется Android API 20 и выше.

%%% Local Variables: debug
%%% mode: latex
%%% TeX-master: "rpz-os"
%%% End: