\Conclusion

	При написании программного обеспечения была проделана работа по изучению литературы, специализирующейся на создании драйверов устройств. Изучены способы беспроводной передачи данных, разновидности датчиков Android и их принцип работы. На базе полученных знаний были проанализированы достоинства и недостатки структур драйверов интерфейсов, датчиков смартфона, таких как акселерометр и гироскоп и передачи данных через bluetooth и wi-fi. Изучены механизмы встраивания драйвера устройства в ядро Linux, создание и дальнейшая работа bluetooth серверов на основе протоколов rfcomm и sdp. 
	
	Программное обеспечение разработано в соответствии с техническим заданием и протестировано на нескольких устройствах.
	
	Данный комплекс программ имеет несколько направлений дальнейшего развития:
\begin{itemize}
\item Добавление фильтрации координат, необходимого для более плавного перемещения курсора.
\item Исключение драйвера и bluetooth сервера путем представления Android устройства HID устройством.
\item Добавление функциональности мыши (центральная кнопка, колесо).
\end{itemize}
Есть место для оптимизации приложения и расширения пользовательского интерфейса.

%%% Local Variables: debug
%%% mode: latex
%%% TeX-master: "rpz-os"
%%% End: 